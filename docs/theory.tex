\documentclass{article}
\usepackage[T2A]{fontenc}
\usepackage[utf8]{inputenc}
\usepackage[english,russian]{babel}
\usepackage[top=2cm,bottom=2cm,left=1.5cm,right=1.5cm]{geometry}
\usepackage{amssymb}
\usepackage{amsmath}
\usepackage{amsfonts}
\usepackage{euscript}

\begin{document}
\section{Почему модифицированный алгоритм Obuchi портит исходную карту меньше других}
\begin{enumerate}
    \label{intro}
    \item Исходные данные есть множество точек $V = \left\{p_1, p_2, ... , p_n\right\}$, на выходе мы хотим получить множество 
    $\delta V = \left\{\delta {p_1}, \delta{p_2}, ... \delta{p_n} \right\}$ смещений исходных точек, 
    в котором запрятался наш watermark. $p_i, \delta{p_i} \in \mathbb{R}^2$. 
    Давайте включим воображение и на секундочку посчитаем, что исходными данными на самом деле является какая-то облать $\Omega \subset \mathbb{R}^2$,
    скажем, выпуклая оболочка $\left\{p_i\right\}$. Теперь представим, что для встраивания watermark'а мы смещаем не только наши точки $\left\{p_i\right\}$, 
    но всю область $\Omega$. То есть, мы-то наивно полагали, что нам надо задать функию $f_d: \left\{p_i\right\} \to \left\{\delta p_i \right\}$, 
    а на самом деле нам надо задать функцию $f_c: \Omega \to \mathbb{R}^2$. Причем функция $f_c$ есть не абы что, а PLIS $f_d$.
    
    \item Для того, чтобы задать PLIS нам нало для начала задаться триангуляцией $V$. А какой именно триангуляцией? 
    Rippa говорит нам, что реальные пацаны используют для подобных целей триангуляцию Делоне, так как для нее погрешность интерполяции минимальна. 
    А что есть согласно Rippa погрешность интерполяции (roughness)? Барабанная дробь -- энергия Дирихле $$E_D(f_c) = \int_{\Omega}{||{\nabla{f_c}}||^2 d\Omega}.$$
    
    \item \label{Criterion}
    Теперь у нас есть триангуляция $\mathfrak{T}$ множества точек $V$ и функция $f_c: \Omega \to \mathbb{R}^2$ . Давайте определимcя, чего же мы хотим от $f_c$.
    Можно, конечно, в качестве $f_c$ взять функцию, переводящую наши данные в надпись ''здесь был \%username\%'', тогда мы всем сможем потом доказать, 
    что это именно наши данные, но как-то многовато мы при этом информации потеряем. То есть, мы хотим минимизировать потерю информации, 
    то есть некоторую функцию от $f_c$. Я утверждаю, что эта самая загадочная функция, которую мы хотим минимизировать есть, барабанная дробь,
    энергия Дирихле $$E_D(f_c) = \int_{\Omega}{||{\nabla{f_c}}||^2 d\Omega}.$$
    
    \item \label{PoithlerTheorem}
    Давайте теперь вернемся с просторов непрерывных небес на нашу бренную дискретныую землю. Итак, мы хотим минимизировать $E_D(f_c)$. 
    Как выразить это в терминах $f_d$? Мудрый Poithler утвержает, что $E_D(f_c)$ при заданной триангуляции $\mathfrak{T}$ выражается как 
        \begin{multline}
        \label{EnergyDiscretization}
            E_D(f_c) = \frac{1}{4} \sum_{\left(i, j, k \right) \in \mathfrak{T}}{\ctg{\alpha_{ij}}||f_i - f_j||^2 + 
            \ctg{\alpha_{jk}}||f_j - f_k||^2 + \ctg{\alpha_{ki}}||f_k - f_i||^2} = \\
            = \frac{1}{2} \sum_{\left( i, j \right) \in E\left(\mathfrak{T}\right)}{w_{ij} ||f_i - f_j||^2} \mbox{, где } w_{ij} = 
            \frac{1}{2}\left(\ctg\alpha_{ij} + \ctg\alpha_{ji}\right).
        \end{multline}
    Точнее утверждает он это про скаляро-значную функцию $f$, но я искренне надеюсь, что это без труда обобщается на векторо-значную.

    Истинная мудрость проявленная Poithler'ом заключается в том, что он доказал формулу~(\ref{EnergyDiscretization}) в куда более общем случае.
    В случае же когда $\Omega$ есть выпуклая оболочка множества точек плоскости и $f_c$ есть PLIS $f_d$ по триангуляции $\mathfrak{T}$ можно посчитать $E_D(f)$ по рабоче-крестьянски.
    \begin{multline*}
        E_D(f) = \frac{1}{2} \int_{\Omega}{||\nabla{f}||^2 d\Omega} = \frac{1}{2} \sum_{T \in \mathfrak{T}}\int_{T}{||\nabla{f|_T}||^2 dT}; 
        \mbox{ } T = \triangle ABC; \mbox{ } A=(x_1, y_1), B=(x_2, y_2), C=(x_3, y_3); \\
        f|_T = f(x, y), f(A) = f_1, f(B) = f_2, f(C) = f_3; \int_{T}{||\nabla{f|_T}||^2 dT} = 
        \int_{T}{\left[\left(\frac{\partial f}{\partial x}\right)^2 + \left(\frac{\partial f}{\partial y}\right)^2 \right] dx dy} \\
    \end{multline*}
    Перейдем к барицентрическим координатам $u, v$: $T = \left\{(u, v)\mbox{ } | \mbox{ } (u, v) \in \left[0, 1 \right]^2 ,\mbox{ } u + v <= 1 \right\}$. 
    \begin{multline*}
        \begin{cases}
            x = x_1 u + x_2 v + (1-u-v) x_3 = (x_1 - x_3) u + (x_2 - x_3) v \\
            y = y_1 u + y_2 v + (1-u-v) y_3 = (y_1 - y_3) u + (y_2 - y_3) v
        \end{cases} 
        \Leftrightarrow
        \begin{pmatrix}
            x_1 - x_3 & x_2 - x_3 \\
            y_1 - y_3 & y_2 - y_3 
        \end{pmatrix} 
        \begin{pmatrix}
            u \\
            v
        \end{pmatrix} 
        =
        \begin{pmatrix}
            x \\
            y
        \end{pmatrix} \\
        \Delta = (x_1 - x_3) (y_2 - y_3) - (x_2 - x_3) (y_2 - y_3) \\ 
        \Delta_u = (y_2 - y_3) x - (x_2 - x_3) y, \mbox{ } \Delta_v = -(y_1 - y_3) x + (x_1 - x_3) y \\
        u = \frac{\Delta_u}{\Delta}, \mbox{ } v = \frac{\Delta_v}{\Delta} \mbox{, } dxdy = |\Delta| dudv \\
        \frac{\partial u}{\partial x} = \frac{1}{\Delta}(y_2 - y_3), \mbox{ } \frac{\partial u}{\partial y} = -\frac{1}{\Delta}(x_2 - x_3) \mbox{; }
        \frac{\partial v}{\partial x} = -\frac{1}{\Delta}(y_1 - y_3), \mbox{ } \frac{\partial v}{\partial y} = \frac{1}{\Delta}(x_1 - x_3) \\
        f(u, v) = f_1 u + f_2 v + f_3(1-u-v) = (f_1 - f_3) u + (f_2 - f_3) v \\
        \frac{\partial f}{\partial u} = f_1 - f_3 \mbox{, } \frac{\partial f}{\partial v} = f_2 - f_3. 
    \end{multline*}
    \begin{multline*}
        \frac{\partial f}{\partial x} = \frac{\partial f}{\partial u} \frac{\partial u}{\partial x} + \frac{\partial f}{\partial v} \frac{\partial v}{\partial x} = 
            \frac{1}{\Delta}(y_2 - y_3)(f_1 - f_3) - \frac{1}{\Delta}(y_1 - y_3)(f_2 - f_3) = \\
            \frac{1}{\Delta}\left[(y_2 - y_3)(f_1 - f_3) - (y_1 - y_3)(f_2 - f_3)\right]. 
    \end{multline*}
    \begin{multline*}
        \frac{\partial f}{\partial y} = \frac{\partial f}{\partial u} \frac{\partial u}{\partial y} + \frac{\partial f}{\partial v} \frac{\partial v}{\partial y} = 
            -\frac{1}{\Delta}(x_2 - y_3)(f_1 - f_3) + \frac{1}{\Delta}(x_1 - x_3)(f_2 - f_3) = \\
            \frac{1}{\Delta}\left[-(x_2 - x_3)(f_1 - f_3) + (x_1 - x_3)(f_2 - f_3)\right]. 
    \end{multline*}
    \begin{multline*}
        \left[\left(\frac{\partial f}{\partial x}\right)^2 + \left(\frac{\partial f}{\partial y}\right)^2 \right] = 
        \frac{1}{\Delta^2} 
        [
            (f_1 - f_3)^2 [(y_2 - y_3)^2 + (x_2 - x_3)^2] + (f_2 - f_3)^2 [(y_1 - y_3)^2 + (x_1 - x_3)^2] - \\
            - 2 (f_1 - f_3)(f_2 - f_3)((y_2 - y_3)(y_1 - y_3) + (x_2 - x_3)(x_1 - x_3))
        ]
    \end{multline*}
    Введем обозначения: $\vec{AB} = \vec{c}$, $\vec{BC} = \vec{a}$, $\vec{CA} = \vec{b}$. Тогда предыдущую формулу можно записать следующим образом.
    \begin{multline*}
        \left[\left(\frac{\partial f}{\partial x}\right)^2 + \left(\frac{\partial f}{\partial y}\right)^2 \right] = 
        \frac{1}{\Delta^2} 
        \left[
            (f_1 - f_3)^2 a^2 + (f_2 - f_3)^2 b^2 - 2 (f_1 - f_3)(f_2 - f_3)(\vec{a}, \vec{b}) 
        \right] = \\
        = \frac{1}{\Delta^2} 
        \left[
            (f_1 - f_3)^2 a^2 + (f_2 - f_3)^2 b^2 - (f_1 - f_3)(f_2 - f_3)(a^2 + b^2 - c^2) 
        \right] = \\
        = \frac{1}{\Delta^2} 
        \left[
            a^2 (f_1 - f_3)^2 + b^2 (f_2 - f_3)^2 - (a^2 + b^2 - c^2)(f_1 f_2 - f_3(f_1 + f_2) + f_3^2)
        \right] = \\
        = \frac{1}{2 \Delta^2} 
        \left[
            2 a^2 (f_1 - f_3)^2 + 2 b^2 (f_2 - f_3)^2 - (a^2 + b^2 - c^2)([2 f_1 f_2 - f_1^2 - f_2 ^ 2] + 
            [f_1 ^2 - 2 f_3 f_1 + f_3 ^ 2] + [f_2^2 - 2 f_3 f_2 + f_3^2])
        \right] = \\
        = \frac{1}{2 \Delta^2} 
        \left[
            2 a^2 (f_1 - f_3)^2 + 2 b^2 (f_2 - f_3)^2 - 
            \left( a^2 + b^2 - c^2 \right) \left(-(f_1 - f_2)^2 + (f_1 - f_3)^2 + (f_2 - f_3)^2 \right)
        \right] = \\
        = \frac{1}{2 \Delta^2} 
        \left[
            (a^2 + b^2 - c^2)(f_1 - f_2)^2 + (a^2 + c^2 - b^2)(f_1 - f_3)^2 + (b^2 + c^2 - a^2)(f_2 - f_3)^2 
        \right] = \\
        = \frac{1}{\Delta^2} 
        \left[
            ab \cos{(\gamma)} (f_1 - f_2)^2 + ac \cos{(\beta)}(f_1 - f_3)^2 + bc \cos{(\alpha)}(f_2 - f_3)^2 
        \right] = \\
        = \frac{2S(T)}{\Delta^2} 
        \left[
            \cot{(\gamma)} (f_1 - f_2)^2 + \cot{(\beta)}(f_1 - f_3)^2 + \cot{(\alpha)}(f_2 - f_3)^2 
        \right].
    \end{multline*}
    \begin{multline*}
        \int_{T}{\left[\left(\frac{\partial f}{\partial x}\right)^2 + \left(\frac{\partial f}{\partial y}\right)^2 \right] dx dy} 
        = \int\limits_0^1\int\limits_0^{1-u} \frac{2S(T)}{\Delta^2} 
        \left[
            \cot{(\gamma)} (f_1 - f_2)^2 + \cot{(\beta)}(f_1 - f_3)^2 + \cot{(\alpha)}(f_2 - f_3)^2 
        \right] 
        |\Delta| du dv = \\
        = \frac{2S(T)}{|\Delta|} 
        \left[
            \cot{(\gamma)} (f_1 - f_2)^2 + \cot{(\beta)}(f_1 - f_3)^2 + \cot{(\alpha)}(f_2 - f_3)^2 
        \right]
        \int\limits_0^1\int\limits_0^{1-u} du dv = \\ 
        = \frac{S(T)}{|\Delta|} 
        \left[
            \cot{(\gamma)} (f_1 - f_2)^2 + \cot{(\beta)}(f_1 - f_3)^2 + \cot{(\alpha)}(f_2 - f_3)^2 
        \right].
    \end{multline*}
    Осталось заметить, что $|\Delta| = 2S(T)$. Окончательно получаем 
    $$
        \int_T ||\nabla{f}||^2 dT = \frac{1}{2} \left[ \cot{(\gamma)} (f_1 - f_2)^2 + \cot{(\beta)}(f_1 - f_3)^2 + \cot{(\alpha)}(f_2 - f_3)^2 \right] 
    $$
    $$ 
        E_D(f) = \frac{1}{4} \sum_{T \in \mathfrak{T}} \left[ \cot{(\gamma)} (f_1 - f_2)^2 + \cot{(\beta)}(f_1 - f_3)^2 + \cot{(\alpha)}(f_2 - f_3)^2 \right].
    $$     

    \item Обозначенному нами в пункте~\ref{Criterion} критерию качества прекрасно удолетворяет $f_d~=~const$. 
    Действительно, $\nabla{f}~=~0 \Rightarrow E_D(f) = 0$, жаль только информации таким образом не закодируешь. А хотелось бы нам закодировать $k$ бит.
    Obuchi предлагает следующее: выберем какие-нибудь линейно независимые $k$ векторов $\left\{ e^i \right\}_{i=1}^k$ и пусть $f_d = \sum_{i=1}^{k}{m_i e^i}$,
    где $m_i \in \left\{ -1, 1 \right\}$. Если $e^i$ ортогональны, то из $f_d$ однозначно восстанавливается watermark: $m_i = (f_d, e_i)$.

    Осталось выбрать $\left\{ e^i \right\}_{i=1}^k$. Утверждается, что в смысле минимизации энергии Дирихле оптимальными будут собственные вектора графа триангуляции соответсвующие наименьшим $k$ собственным числам. Для доказательства этого нам на помощь приходит Elisa Ross.
    
    \item \label{RossTheorem}
    Пусть задан граф $G = (V, E, w), w: E \to \mathbb{R}$ и функция $\rho: V(G) \to \mathbb{R}^m$. Введем обозначения 
    $$ w(e = (u, v)) = w_{uv},$$
    $$ n = |V|, $$
    $$ R = \left(\rho(v_1), \rho(v_2), ... , \rho(v_n)\right), \dim{R} = n \times m, $$
    $$ \EuScript{E}\left(\rho\right) = \sum_{\left(u, v \right) \in E\left( G \right)}{w_{uv} ||\rho(u) - \rho(v)||^2}.$$
    $$ L, \dim{L} = n \times n, L_{uv} = \begin{cases}
        \sum_{(v, x) \in E(G)}{w_{vx}},&\text{$u = v;$} \\
        -w_{uv},&\text{$u \sim v;$} \\
        0, &\text{otherwise.}
    \end{cases}
    $$
    
    Заметим, что $\EuScript{E}(\rho) = tr(R^T L R)$. В предположении, что столбцы $R$ линейно независимы и нормированы, $\min{\EuScript{E}(\rho)}$ достигается, 
    когда $R$ составлена из векторов, натянутых на собственные вектора $L$ соответсвующие наименшьшим $m$ собственным числам $L$.
     
    \item Применим утверждение пункта~\ref{RossTheorem}. Пусть $G$ -- граф триангуляции, $V(G)$ -- то же множество, что и в пункте~\ref{intro}, 
    веса ребер $w_{uv}$ мы берем те же, что и в пункте~\ref{PoithlerTheorem}. $\rho(v) = \left( {(m_1 e^1)_v}, {(m_2 e^2)_v}, ..., {(m_k e^k})_v \right)$.
    $$
        \EuScript{E}(\rho) = \sum_{\left( u, v \right) \in E \left( G \right)}{w_{uv} ||\rho(u) - \rho(v)||^2} = 
        \sum_{\left( u, v \right) \in E \left( G \right)}{w_{uv} \sum_{i=1}^k{\left((m_i e^i)_v - (m_i e^i)_u \right)^2}} =
        \sum_{\left( u, v \right) \in E \left( G \right)}{w_{uv} ||f_d(u) - f_d(v)||^2} = E_D(f)
    $$
    То есть, для минимизации энергии энергии Дирихле, надо взять $f_d = \sum_{i=1}^k{m_i e^i}$, где $e^i$ -- собственные вектора матрицы $L$ определенной 
    в пункте~\ref{RossTheorem}.
\end{enumerate}

\end{document}
