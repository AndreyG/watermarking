\startprefacepage

Цифровые водяные знаки (ЦВЗ) --- технология, созданная для защиты авторских прав мультимедийных файлов,
однако в настоящее время этот термин принято понимать в более широком смысле, а именно, как технологию
внедрения некоторой информации в цифровые данные. В частности, в данной работе нас будет интересовать
внедрение ЦВЗ в данные географических информационных систем (ГИС).

ГИС оперируют геометрическими примитивами (точки, полилинии, контуры), с аттрибутами 
произвольного типа. В общем случае для внедрения можно использовать только геометрические данные,
то есть задача создания ЦВЗ для ГИС-данных является частным случаем задачи создания ЦВЗ для двумерных векторных данных.

Жизненный цикл ЦВЗ состоит из следующих фаз:
\begin{itemize}
  \item модификация исходных данных (внедрение водяных знаков) для добавления к данным 
  некоторого сообщения, скажем, фамилии автора или названия компании-владельца; 
  \item атаки модифицированных данных злоумышленником, то есть изменения их с целью уничтожения 
  или изменения водяных знаков;
  \item извлечение водяных знаков из атакованных данных.
\end{itemize}

Основными требованиями, предъявляемыми к алгоритмам внедрения ЦВЗ являются:
\begin{itemize}
  \item устойчивость к атакам --- способность извлекать внедренные ЦВЗ из атакованных данных;
  \item незначительность искажения исходных данных при внедрении ЦВЗ.
\end{itemize}

Очевидно, что эти требования конфликтуют, и обычно при разработке алгоритма внедрения ЦВЗ
больше внимания уделяется первому из них. Цель данной работы --- разработка критерия степени искажения
ГИС-данных и разработка устойчивого к атакам алгоритма внедрения ЦВЗ
минимизирующего предложенный критерий.

