\startprefacepage

Цифровые водяные знаки (ЦВЗ) --- технология созданная для защиты авторских прав мультимедийных файлов,
однако в настоящее время этот термин принято понимать в более широком смысле, а именно, как технологию
внедрения некоторой информации в цифровые данные. В частности, в данной работе нас будет интересовать
внедрение ЦВЗ в данные географических информационных систме (ГИС).

ГИС оперируют геометрическими примитивами (точки, полилинии, контура), с аттрибутами 
произвольного типа. В общем случае для внедрения можно использовать только геометрические данные,
то есть задача создания ЦВЗ для ГИС эквивалентна задаче создания ЦВЗ для двумерных векторных данных.

Жизненный цикл ЦВЗ состоит из следующих фаз.
\begin{itemize}
  \item Модификация исходных данных (внедрение водяных знаков) для добавления к данным 
  некоторого сообщения, скажем, фамилии автора или названия компании-владельца. 
  На этом этапе может применяется секретный ключ. 
  \item Атаки модифицированных данных злоумышленником, то есть изменения их с целью уничтожения 
  или изменения водяных знаков.
  \item Извлечение водяных знаков из атакованных данных, для доказательства авторских прав.
\end{itemize}

Основными требованиями, предъвляемыми к алгоритмам внедрения ЦВЗ являются:
\begin{enumerate}
  \item устойчивость к атакам --- способность извлекать внедренные ЦВЗ из атакованных данных;
  \item незначительность искажения исходных данных при внедрении.
\end{enumerate}

Очевидно, что эти требования конфликтуют, и обычно при разработке алгоритма внедрения ЦВЗ
больше внимания уделяется первому из них. Цель данной работы --- разработка численной меры искажения
двуменых векторных данных при внедрении ЦВЗ и модификация одного из известных алгоритмов
для минимизации предложенной меры.

