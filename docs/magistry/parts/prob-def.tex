\section{Формальная постановка задачи}
\label{sec:prob_def}

ГИС-данные --- это планарный прямолинейный граф $G = (V, E)$, обладающий следующими свойствами:
\begin{itemize}
    \item кластеризованность;
    \item равномерное распределение вершин внутри кластера.
\end{itemize}
Кроме того, для ГИС-данных характерен масштаб и допустимая погрешность.

В дальнейшем вместо термина ``граф'' будет часто употребляться термин ``карта''. Будем также использовать термин 
\textit{оригинальная} для исходной карты, термин \textit{подписанная} для карты с внедренными 
ЦВЗ и термин \textit{атакованная} для подписанной карты, подвергшейся атаке злоумышленника.

Подписанная карта может подвергаться следующим видам атак.
\begin{description}
    \item[Переупорядочивание.] Переупорядочивание данных злоумышленником исключает возможность 
    использования порядка на вершинах или ребрах графа для внедрения ЦВЗ.
    \item[Преобразование подобия.] Злоумышленник может применить движение плоскости или 
    равномерное масштабирование множества точек карты без какого-либо ущерба для визуального качества. 
    \item[Вставка и удаление вершин.] Это возможно при упрощении или, наоборот, сглаживании конутров и полилиний. 
    Очевидно, алгоритм внедрения ЦВЗ, способный противостоять данной атаке, не может рассчитывать даже на то, что количество вершин в исходной и атакованной карты совпадает.
    \item[Шум.] Применение аддитивного шума к координатам вершин. Некоторые авторы \cite{Kim, Shao, Bazin} считают, 
    что требования, предъявляемые к точности карт, используемых в ГИС, столь высоки, что они не позволяют злоумышленникам атаковать карты таким образом.
    Соответственно, разработанные этими авторами алгоритмы неустойчивы к добавлению шума.
    \item[Обрезка (cropping).] Это самая ``сильная'' атака, состоящая в удалении куска карты. 
    Только один из известных автору алгоритм (\cite{Ohbuchi}) частично устойчив к ней.
\end{description}  

Очевидно, что алгоритмы, устойчивые хотя бы к первым трем из перечисленных видов атак, должные изменять координаты вершин графа, 
но при этом нет смысла добавлять/удалять вершины, или менять их порядок. 
Кроме того, существует важное требование, предъявляемое алгоритмам внедрения ЦВЗ --- сохранение топологии графа, из которого, в частности, следует,
что множество ребер не должно изменяться.
Таким образом, задача решаемая в данной работе, формулируется следующим образом. 
\begin{itemize}
    \item Защитить авторские права на ГИС-данные, изменяя координаты вершин в пределах допустимой погрешности. 
    Изменение координат вершин не должно нарушать топологии графа.
    \item Сформулировать критерий $K$ степени искажения исходных данных.
\end{itemize}
Или более формально.
\begin{itemize}
    \item Найти функцию $f(V_G, key, msg) \to V_G'$, такую что существует обратная функция $\exists f^{-1}(g(V_G'), V_G, key) \to msg$, где
    \begin{itemize}
        \item $key$ --- секретный ключ;
        \item $msg$ --- внедряемое сообщение, удостоверяющее авторство;
        \item $g: V_G' \to V_G''$ --- атака злоумышленником.
    \end{itemize}
    \item Определить и минимизировать функцию $K(G, V_G') \to \mathbb{R}$.
\end{itemize}
