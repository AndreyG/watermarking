\section{Постановка задачи}
\label{sec:prob_def}

Входными данными алгоритма внедрения ЦВЗ в двумерные векторные данные является планарный прямолинейный граф $G = (V, E)$. 
Поскольку, когда говорят о ЦВЗ для двумерных векторных данных, как правило, речь идет о данных используемых ГИС 
вместо термина ``граф'' часто будет употребляться термин ``карта''. Будем также использовать термин 
``оригинальная'' для исходной карты, термин ``подписанная'' для карты с внедренными ЦВЗ и термин 
``атакованная'' для подписанной карты подвергшейся атаке злоумышленника.
Карта состоит из набора изолированных вершин, простых полилиний и контуров.
Вершины графа заданы координатами в некоторой системе.

Возможны следующие виды атак.
\begin{description}
    \item[Переупорядочивание.] Возможное переупорядочивание данных злоумышленником исключает возможнсть 
    использования порядка на вершинах или ребрах графа для внедрения ЦВЗ.
    \item[Преобразование подобия.] Злоумышленник может к применить движения плоскости или 
    равномерное масштабирование множества точек карты без какого-либо ущерба для визуального качества. 
    \item[Вставка и удаление вершин.] Это возможно при упрощении или наоборот сглаживании конутров и полилиний. 
    Очевидно, алгоритм внедрения ЦВЗ, способный противостоять данной атаке, не может рассчитывать даже на то, что количество вершин в исходной и атакованной карты совпадает.
    \item[Шум.] Применение аддитивного случайного шума к координатам вершин. Некоторые авторы \cite{Kim, Shao, Bazin} считают, 
    что требования предъявляемые к точности карт используемых в ГИС столь высоки, что они не позволяют злоумышленникам использовать данный вид атак,
    соответственно, разработанные данными авторами алгоритмы неустойчивы к нему.
    \item[Обрезка (cropping).] Это самая ``сильная'' атака, только один из известных автору алгоритм (\cite{Ohbuchi}) частично устойчив к ней.
\end{description}  

Очевидно, что алгоритмы, устойчивые к хотя бы первым трем из перечисленных видов атак, должные изменять координаты вершин графа, но при этом нет смысла добавлять/удалять вершины, или менять их порядок. 
Из необходимости сохранения топологии графа следует, что множество ребер при этом не должно изменяться.
Таким образом алгоритм внедрения ЦВЗ представляет собой фукнцию $f: (V, message, key) \to V', |V| = |V'|$,
бьюущую из множества троек (вершины, внедряемое сообщение, секретный ключ) в множество модифицированных вершин.
