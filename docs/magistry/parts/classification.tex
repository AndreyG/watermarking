\section{Классификация алгоритмов внедрения ЦВЗ}
\label{sec:classification}
Все алгоритмы внедрения ЦВЗ в двумерные векторные данные можно, грубо говоря, поделить на две категории: 
работающие в ``пространственной'' области, и работающие в ``частотной'' области. 
Алгоритмы, принадлежащие первой категории, изменяюют непосредственно 
координаты вершин \cite{Voight, Kim, Chang, Bazin}. Алгоритмы из второй категории вычисляют некоторые 
взаимно-однозначно связанные с координатами вершин коэффициенты (дискретное преобразование Фурье, 
дискретное косинусное преобразование, дискретные вейвлет-преобразования и т.д.), изменяют их, а затем 
преобразуют обратно в уже измененные координаты \cite{Ohbuchi, Ohbuchi3D, Praun}.  
Сильной стороной алгоритмов, принадлежащих первой категории, является возможность контролированть искажение
исходных данных при внедрении ЦВЗ, однако алгоритмы из второй категории более устойчивы к атакам.

Другим признаком, по которому можно классифицировать алгоритмы внедрения, является использование 
оригинальной карты для извлечения ЦВЗ. Те алгоритмы, которые требуют оригинальной карты для извлечения ЦВЗ,
называются хорошо информированными (well-informed), в противоположность им не требующие называются ослепленными
(blinded). Ослепленность есть весьма полезное свойство в случае больших карт производимых на лету, 
так как она избавляет от необходимости хранить оригинальные версии, однако, очевидно, она уменьшает возможности
извлечения, а значит, и внедрения ЦВЗ.
