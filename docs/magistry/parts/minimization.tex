\section{Модификация базового алгоритма}
\label{sec:minimization}

Если $\mathbf{f} = \alpha \left[ p_1 \mathbf{e}_1 + p_2 \mathbf{e}_2 + \dots + p_k \mathbf{e}_k \right],$ 
где $\alpha$ --- некоторая положительная константа, $p_i \in \left\{-1, 1 \right\}$ и 
$\{\mathbf{e}_i\}_{i=1}^k$ --- ортонормированная система векторов, то 
$E_C(f) = \frac{1}{4}\conjugate{\mathbf{f}^T} L'' \mathbf{f}$ будет минимизировано, 
когда $\{\mathbf{e}_i\}_{i=1}^k$ суть собственные вектора $L''$ соответствующии $k$ наименьшим 
собственным числам. 

Базовый алгоритм (\ref{sec:base}) строил вектор $\mathbf{g}$ разниц между новыми и старыми
координатми вершин (формула (\ref{formula:g})), а $E_C$ --- функционал применяемый к $f$, 
$f(v) = v + g(v)$. Здесь нет противоречия, так как $$E_C(g) = E_C(f - v) = E_C(f),$$ ибо
функционал $E_C$ линейный и тождественная функция $id(v) = v$ конформна. 

Таким образом, чтобы минимизировать конформную энергию при использовании базового алгоритма 
необходимо выбрать вектора $\{\mathbf{h}_i\}_{i=1}^k$ из формулы (\ref{formula:g}) равными собстенным векторам $L''$, соответсвующим наименьшим собственным числам.

Конформная энергия была определена (\ref{sec:continious}) как мера искажения локальных углов.
Можно посмотреть, что выражает $E_C(f)$ в терминах углов триангуляции $\mathfrak{T}$. 
\begin{multline*}
  E_C(f) = E_D(f) - \int_{\Omega} \det (\nabla f) d\Omega = \\ 
  = \frac{1}{4} \sum_{v_i v_j v_k \in \mathfrak{T}} \cot{\alpha_{ij}}|f_i - f_j|^2 + 
  \cot{\alpha_{jk}}|f_j - f_k|^2 + \cot{\alpha_{ki}}|f_k - f_i|^2 - 4S(f_i f_j f_k),
\end{multline*}
где $S(f_i f_j f_k)$ есть ориентированная площать треугольника $\triangle f_i f_j f_k$. 
В предположении, что функция $f$ не сильно искажает область $\Omega$, 
ориентация треугольника $\triangle f_i f_j f_k$ равна ориентации треугольника 
$\triangle v_i v_j v_k$, то есть треугольник $\triangle f_i f_j f_k$ ориентирован против часовой
стрелки,
\begin{multline*}
  4 S(f_i f_j f_k) = \cot{\alpha_{ij}'}|f_i - f_j|^2 + \cot{\alpha_{jk}'}|f_j - f_k|^2 + \cot{\alpha_{ki}'}|f_k - f_i|^2,
\end{multline*}
где $\alpha_{ij}'$, $\alpha_{jk}'$, $\alpha_{ki}'$ --- углы, противолежащие сторонам $f_i f_j$, 
$f_j f_k$, $f_k f_i$ треугольника $\triangle f_i f_j f_k$ соответственно. 
Таким образом, при определенных предположениях
\begin{multline*}
  E_C(f) = \frac{1}{4} \sum_{v_i v_j v_k \in \mathfrak{T}} (\cot{\alpha_{ij}} - \cot{\alpha_{ij}'})|f_i - f_j|^2 + \\
  + (\cot{\alpha_{jk}} - \cot{\alpha_{jk}'})|f_j - f_k|^2 + (\cot{\alpha_{ki}} - \cot{\alpha_{ki}'})|f_k - f_i|^2,
\end{multline*} откуда видно, что подобные треугольники перешедшие при внедрении ЦВЗ в подобные,
дают вклад в меру искажения пропорциональный своей площади, что довольно естественно.

До этого момента на триангуляцию $\mathfrak{T}$ не накладывалось никаких условий, но очевидно, что
$E_C$ зависит от $\mathfrak{T}$. Во-первых, было доказано, что минимизация $E_C$ влечет 
минимизацию углов триангуляции в целом, но наиболее важны с точки зрения визуального качества
карты угла между смежными ребрами исходных контуров и полилиний. Поэтому логично, 
добавить исходные контура и полилинии в триангуляцию $\mathfrak{T}$ как ограничения.
Во-вторых, при переходе от дискретного к непрерывному случаю триангуляция была использована для
построения кусочно-линейной интерполяционной поверхности, поэтому она должна быть выбрана так,
чтобы минимизировать погрешность интерполяции. Согласно теореме Rippa (\cite{Rippa}), 
триангуляция Делоне доставляет наименьшую погрешность КЛИП среди всех триангуляций 
множества точек.
Но так как для доказательства теоремы Rippa необходим только локальный критерий 
пустого круга(\cite{Chen}), она легко обобщается и на случай триангуляций с ограничениями

Таким образом, при построении сетки для вычисления частотного представления карты в 
базовом алгоритме (\ref{sec:base}) действительно имеет смысл строить триангуляцию Делоне 
с ограничениями. 
