\section{Непрерывный случай}
\label{sec:continious}

Рассмотрим некоторую односвязную открытую область $\Omega \subset \mathbb{R}^2$ и функцию 
$f: \Omega \to \mathbb{R}^2$, $f = \begin{pmatrix} f_x(x, y) \\ f_y(x, y) \end{pmatrix}$. 
Предположим, что функция $f$ дифференцируема, а значит сущетсвуют частные производные $f_x, f_y$ по $x$ и $y$. 
Будем использовать следующие обозначения
$$
  \nabla f_x = \begin{pmatrix} \frac{\partial f_x}{\partial x} \\ \frac{\partial f_x}{\partial y} \end{pmatrix}, 
  \nabla f_y = \begin{pmatrix} \frac{\partial f_y}{\partial x} \\ \frac{\partial f_y}{\partial y} \end{pmatrix}, 
  \nabla f = \begin{pmatrix} 
    \frac{\partial f_x}{\partial x} & \frac{\partial f_x}{\partial y} \\
    \frac{\partial f_y}{\partial x} & \frac{\partial f_y}{\partial y} \\
  \end{pmatrix}. 
$$
Оценим изменения локальных углов после применения функции $f$.
Пусть $\vect{r}~=~\begin{pmatrix} x \\ y \end{pmatrix}$ произвольная точка, 
$\vect{r_1} = \vect{r} + \vect{dr} = \begin{pmatrix} x + dx \\ y + dy \end{pmatrix}$ есть точка смещенная относительно $\vect{r}$ на произвольный малый вектор $\vect{dr}$
\begin{figure}[htb]
  \begin{minipage}[b]{1.0\linewidth}
    \centering
    \begin{tikzpicture} 
      \coordinate[label=below:$r$]        (origin)      at (0, 0);
      \coordinate[label=above:$r_1$]   (dxdy)        at origin + (30:2);
      \coordinate[label=above:$r_2$]  (dxdyrot)     at origin + (100:2);

      \draw[->] (origin)--(dxdy);
      \draw[->] (origin)--(dxdyrot);

      \draw (origin) -- (30:0.5) arc (30:100:0.5);
      \draw (60:0.74) node {$\alpha$};
  
      \draw [help lines][ ->] (2.5, 1) -- (5.0, 1);

      \begin{scope}[shift={(0:8.5cm)}]
        \coordinate[label=below:$f(r)$]   (forigin)      at (0, 0);
        \coordinate[label=above:$f(r_1)$] (fdxdy)        at ($(forigin) + (45:1.9)$);
        \coordinate[label=above:$f(r_2)$] (fdxdyrot)     at ($(forigin) + (100:2.1)$);
        \coordinate[label=left:$f(r) + R^\alpha df$] (rotfdxdy)    at ($(forigin) + (115:1.9)$);

        \draw[->] (forigin)--(fdxdy);
        \draw[->] (forigin)--(fdxdyrot);
        \draw[->][dashed] (forigin)--(rotfdxdy);
        \draw[very thick] (rotfdxdy)--(fdxdyrot);
        \draw (forigin) -- (45:0.5) arc (45:115:0.5);
        \draw (80:0.74) node {$\alpha$};
      \end{scope}
    \end{tikzpicture} 
  \end{minipage}
  
  \caption{Изменение локальных углов}
  \label{fig:local_angles}
\end{figure}
и $\vec{r_2}$ есть образ точки $\vec{r_1}$ при повороте относительно точки $\vec{r}$ 
на произвольный угол~$\alpha$,
$$
\vect{r_2} = \vect{r} + R^\alpha \vect{dr} = \begin{pmatrix} x \\ y \end{pmatrix} + 
\begin{pmatrix} \cos \alpha & -\sin \alpha \\ \sin \alpha & \cos \alpha \end{pmatrix} \begin{pmatrix} dx \\ dy \end{pmatrix},
$$
где $R^\alpha$ --- оператор поворота на угол $\alpha$ (рис. \ref{fig:local_angles}). Эти точки будут переведены
функцией $f$ в точки 
\begin{multline*}
  \vect{f(r)}, 
  \vect{f(r_1)} = \vect{f(r + dr)} = \vect{f(r)} + \vect{df} \approx \vect{f(r)} + (\nabla f) \vect{dr}, \\
  \vect{f(r_2)} = \vect{f(r + R^\alpha dr)} ~\approx \vect{f(r)} + (\nabla f)(R^\alpha \vect{dr}).
\end{multline*}
соответственно. Если $f$ сохраняет локальные углы, то точка $\vect{f(r_2)}$ должны быть равна точке 
$\vect{f(r_1)}$ повернутой относительно точки~$\vect{f(r)}$ на угол $\alpha$. 
Иначе мы можем использовать длину вектора $\vect{q(r, dr, \alpha)}$
$$\vect{q(r, dr, \alpha)} = \left(\vect{f(r)} + R^\alpha \vect{df}\right) - \vect{f(r_2)}$$ 
как меру изменения угла $\alpha$ в точке $r$ при действии функции $f$.
\begin{multline}
\label{formula:angle_change_vec}
  \vect{q(r, dr, \alpha)} = \left(\vect{f(r)} + R^\alpha \vect{df} \right) - \vect{f(r_2)} \approx \\
  \approx (\vect{f(r)} + R^\alpha (\nabla f) \vect{dr}) - (\vect{f(r)} + (\nabla f)(R^\alpha \vect{dr})) =  
  \left[R^\alpha (\nabla f) - (\nabla f) R^\alpha \right] \vect{dr}. 
\end{multline}
\begin{multline}
\label{formula:ugly_commutant}
  R^\alpha (\nabla f) - (\nabla f) R^\alpha =\
  \begin{pmatrix} 
    \cos \alpha & -\sin \alpha \\ 
    \sin \alpha & \cos \alpha 
  \end{pmatrix} 
  \begin{pmatrix} 
    \frac{\partial f_x}{\partial x} & \frac{\partial f_x}{\partial y} \\
    \frac{\partial f_y}{\partial x} & \frac{\partial f_y}{\partial y} 
  \end{pmatrix} - \\
  - \begin{pmatrix} 
    \frac{\partial f_x}{\partial x} & \frac{\partial f_x}{\partial y} \\
    \frac{\partial f_y}{\partial x} & \frac{\partial f_y}{\partial y} 
  \end{pmatrix}
  \begin{pmatrix} 
    \cos \alpha & -\sin \alpha \\
    \sin \alpha & \cos \alpha 
  \end{pmatrix} = \\
  = \begin{pmatrix}
    \frac{\partial f_x}{\partial x} \cos \alpha - \frac{\partial f_y}{\partial x} \sin \alpha &
    \frac{\partial f_x}{\partial y} \cos \alpha - \frac{\partial f_y}{\partial y} \sin \alpha \\
    \frac{\partial f_x}{\partial x} \sin \alpha + \frac{\partial f_y}{\partial x} \cos \alpha &
    \frac{\partial f_x}{\partial y} \sin \alpha + \frac{\partial f_y}{\partial y} \cos \alpha 
  \end{pmatrix} - \\
  - \begin{pmatrix} 
    \frac{\partial f_x}{\partial x} \cos \alpha + \frac{\partial f_x}{\partial y} \sin \alpha &
    - \frac{\partial f_x}{\partial x} \sin \alpha + \frac{\partial f_x}{\partial y} \cos \alpha \\
    \frac{\partial f_y}{\partial x} \cos \alpha + \frac{\partial f_y}{\partial y} \sin \alpha &
    - \frac{\partial f_y}{\partial x} \sin \alpha + \frac{\partial f_y}{\partial y} \cos \alpha 
  \end{pmatrix} = \\
  = \sin \alpha \begin{pmatrix}
    -(\frac{\partial f_x}{\partial y} + \frac{\partial f_y}{\partial x}) &
    \frac{\partial f_x}{\partial x} - \frac{\partial f_y}{\partial y} \\
    \frac{\partial f_x}{\partial x} - \frac{\partial f_y}{\partial y} &
    \frac{\partial f_x}{\partial y} + \frac{\partial f_y}{\partial x} 
  \end{pmatrix}.
\end{multline}
Подставляя (\ref{formula:ugly_commutant}) в формулу (\ref{formula:angle_change_vec}) получаем
\begin{multline*} 
  \left| \vect{q(r, dr, \alpha)} \right| = 
  \left| \left[R^\alpha (\nabla f) - (\nabla f) R^\alpha \right] \vect{dr} \right| 
  = \left| \sin \alpha \begin{pmatrix}
    -(\frac{\partial f_x}{\partial y} + \frac{\partial f_y}{\partial x}) &
    \frac{\partial f_x}{\partial x} - \frac{\partial f_y}{\partial y} \\
    \frac{\partial f_x}{\partial x} - \frac{\partial f_y}{\partial y} &
    \frac{\partial f_x}{\partial y} + \frac{\partial f_y}{\partial x} 
  \end{pmatrix} \vect{dr} \right| = \\
  = |\sin \alpha| \sqrt{\left(\frac{\partial f_x}{\partial y} + \frac{\partial f_y}{\partial x}\right)^2 + \left(\frac{\partial f_x}{\partial x} - \frac{\partial f_y}{\partial y}\right)^2} \left| dr \right|.
\end{multline*}
Легко видеть, что вектор $\vect{q(r, dr, \alpha)}$ обращается в 0 тогда и только тогда, когда
$$\begin{cases}
  \frac{\partial f_x}{\partial x} = \frac{\partial f_y}{\partial y} \\
  \frac{\partial f_x}{\partial y} = -\frac{\partial f_y}{\partial x} 
\end{cases}$$
то есть функции $f_x, f_y$ удолетворяют уравнениям Коши-Римана, а значит $f$ --- голоморфна. 
С другой стороны, если производная $f$ в смысле $\mathbb{C}$ в точке $\vect{r}$ обращается в 0, это значит, что
$f$ переводит некоторую область карты в одну точку, то есть сильно искажает. Таким образом ''идеальная'' в
смысле сохранения локальных углов функция должны быть голоморфна и ее производная в смысле $\mathbb{C}$ должна
быть отлична от 0, то есть функция должна быть конформна, что естественно.

Если $f$ не конформна длина вектора $\vect{q(r, dr, \alpha)}$ максимальна когда $\alpha = \pm \frac{\pi}{2}$ 
и не зависит от направления $\vect{dr}$. 
\begin{equation*}
  \left(\frac{\left| {q(r, dr, \pm \frac{\pi}{2})} \right|}{|dr|}\right)^2 = 
  \left(\frac{\partial f_x}{\partial y} + \frac{\partial f_y}{\partial x}\right)^2 + 
  \left(\frac{\partial f_x}{\partial x} - \frac{\partial f_y}{\partial y}\right)^2 
\end{equation*}
Предлагается использовать функционал 
\begin{equation}
\label{formula:EC_Def}
  E_C(f) = \frac{1}{2} \int_{\Omega}\left[{ 
    \left(\frac{\partial f_x}{\partial x} - \frac{\partial f_y}{\partial y}\right)^2 + 
    \left(\frac{\partial f_x}{\partial y} + \frac{\partial f_y}{\partial x}\right)^2} \right] d\Omega   
\end{equation}
называемый \textit{конформной энергией} (\cite{Polthier}) в качестве меры искажения области $\Omega$ после
применения к ней функции $f$. 

Формулу~(\ref{formula:EC_Def}) удобно переписать в виде
\begin{multline}
\label{formula:EC}
  E_C(f) = \frac{1}{2} \int_{\Omega}\left[{
      \left(\frac{\partial f_x}{\partial x} - \frac{\partial f_y}{\partial y}\right)^2 + 
      \left(\frac{\partial f_x}{\partial y} + \frac{\partial f_y}{\partial x}\right)^2}\right] d\Omega = \\
  = \frac{1}{2} \int_{\Omega}\left[\left(\frac{\partial f_x}{\partial x}\right)^2 + \left(\frac{\partial f_y}{\partial y}\right)^2 + \left(\frac{\partial f_x}{\partial y}\right)^2 + 
  \left(\frac{\partial f_y}{\partial x}\right)^2 \right] d\Omega - \\
  - \int_{\Omega}\left[\frac{\partial f_x}{\partial x} \frac{\partial f_y}{\partial y} - \frac{\partial f_x}{\partial y} \frac{\partial f_y}{\partial x}\right] d\Omega = \\
  = \frac{1}{2} \int_{\Omega} \left[ \left| \nabla f_x \right| ^ 2 + \left| \nabla f_y \right| ^ 2 \right] d\Omega - \int_{\Omega} \det {\begin{pmatrix} 
    \frac{\partial f_x}{\partial x} & \frac{\partial f_x}{\partial y} \\
    \frac{\partial f_y}{\partial x} & \frac{\partial f_y}{\partial y} \\
  \end{pmatrix}} d\Omega = \\
  = E_D(f) - \int_{\Omega} \det (\nabla f) d\Omega,
\end{multline}
где $E_D(f)$ обозначает фукнционал, называемый \textit{энергией Дирихле}.
