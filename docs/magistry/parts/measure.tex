\chapter{Оценка искажения}

Существует два аспекта искажения исходных данных при внедрении ЦВЗ. Это, во-первых, нарушение топологии графа,
то есть возникновения пересечений между отрезками подписанной карты не только по концам. Во-вторых, искажение
карты определяется ухудшением ее визуального качества.

Первому аспекту посвящена работа \cite{Huber}. В ней предлагается довольно естественная идея ограничить 
возможные смещения вершин областями определяемыми обобщенной диаграммой Вороного (\cite{Held}), что, очевидно,
гаранитирует отсутствие пересечений в подписанной карте.

Визуальное качество карты понятие плохо формализованное. Попытке все-таки оценить его численно посвящена данная
глава. Очевидно, преобразования подобия не искажают карту, а преобразования подобия это все те преобразования, 
которые сохраняют углы при вершинах геометрических 
примитивов, составляющих карту. Поэтому естественно предположить, что искажение карты определяется изменениями
углов между смежными ребрами исходного графа.

Оригинальная и подписанная карты могут быть рассмотрены как связные области на плоскости, поэтому можно сначала
изложить предлагаемые идеи в непрерывном случае, что интуитивно понятнее,  (\ref{sec:continious}), а затем вернуться к 
дискретному~(\ref{sec:discrete}). Затем будет показано~(\ref{sec:minimization}), как модифицировать 
базовый алгоритм~(\ref{sec:base}), для минимизации предлагаемой в данной главе меры искажения.

\section{Непрерывный случай}
\label{sec:continious}

\section{Дискретный случай}
\label{sec:discrete}

Рассмотрим множества sets $V = \left\{v_1, v_2, ... , v_n\right\}, 
V' = \left\{v_1', v_2', ... , v_n'\right\}$, $v_i, v_i' \in \mathbb{R}^2$, $|V| = |V'| = n$. 
Пусть функция $f: V \to V'$ отображает множество $V$ на множество $V'$, $f(v_i) = v_i'$, $1 \le i \le n$. 
Пусть $\Omega$ обозначает выпуклую оболочку $V$. 
Можно построить кусочно-линейную интерполяционную поверхность (КЛИП) значений функции 
$\left\{f(v_i)\right\}_{i=1}^n$ используя какую-либо триангуляцию $\mathfrak{T}$ множества $V$, 
таким образом можно смотреть на $f$ как на функцию бьющую из $\Omega$ в $\mathbb{R}^2$. 
Если точка $v$ лежит в треугольнике $\triangle T = (v_i v_j v_k)$, $\triangle T \in \mathfrak{T}$; $
v = \alpha v_i + \beta v_j + \gamma v_k$, $0 \le \alpha, \beta, \gamma \le 1$, $\alpha + \beta + \gamma = 1$, 
то $f(v) = \alpha f(v_i) + \beta f(v_j) + \gamma f(v_k)$.

Известным фактом (\cite{Pinkall93}) является, что если $f$ --- КЛИП значений 
$\left\{f_i = f(v_i)\right\}_{i=1}^n$ по триангуляции $\mathfrak{T}$, то
\begin{equation*}
  E_D(f) = \frac{1}{4} \sum_{\triangle (i, j, k) \in \mathfrak{T}} \cot{\alpha_{ij}}|f_i - f_j|^2 
  + \cot{\alpha_{jk}}|f_j - f_k|^2 + \cot{\alpha_{ki}}|f_k - f_i|^2,
\end{equation*}
где $\alpha_{ij}$ --- угол противоположный ребру $(v_i, v_j)$ в треугольнике $\triangle T = (v_i, v_j, v_k)$.   

Однако в работе \cite{Pinkall93} заявляется, что очевидно, что верен куда более общий факт, но доказательства не
приводится. Докажем его для одного треугольника $T = \triangle ABC$ и функции $f: T \to \mathbb{R}$ линейной
на нем. Заметим, что это практически то, что нам и требуется, так как $E_D(f) = E_D(f_x) + E_D(f_y)$. 

Пусть $A=(x_1, y_1), B=(x_2, y_2), C=(x_3, y_3)$; $$E_D(f) = \int_{T}{||\nabla{f}||^2 dT} = 
        \int_{T}{\left[\left(\frac{\partial f}{\partial x}\right)^2 + 
                       \left(\frac{\partial f}{\partial y}\right)^2 \right] dx dy}.$$ 

Перейдем к барицентрическим координатам $u, v$: $$T = \left\{(u, v)\mbox{ } | \mbox{ } (u, v) \in \left[0, 1 \right]^2 ,\mbox{ } u + v <= 1 \right\}.$$ 
Введя обозначения \begin{multline*} \Delta = (x_1 - x_3) (y_2 - y_3) - (x_2 - x_3) (y_2 - y_3), \\ \Delta_u = (y_2 - y_3) x - (x_2 - x_3) y, \mbox{ } \Delta_v = -(y_1 - y_3) x + (x_1 - x_3) y, 
\end{multline*}
Легко вывести
\begin{multline*} u = \frac{\Delta_u}{\Delta}, \mbox{ } v = \frac{\Delta_v}{\Delta} \mbox{, } dxdy = |\Delta| dudv; \\
        \frac{\partial u}{\partial x} = \frac{1}{\Delta}(y_2 - y_3), \mbox{ } \frac{\partial u}{\partial y} = -\frac{1}{\Delta}(x_2 - x_3) \mbox{; }
        \frac{\partial v}{\partial x} = -\frac{1}{\Delta}(y_1 - y_3), \mbox{ } \frac{\partial v}{\partial y} = \frac{1}{\Delta}(x_1 - x_3); \\
        f(u, v) = f_1 u + f_2 v + f_3(1-u-v) = (f_1 - f_3) u + (f_2 - f_3) v; \\
        \frac{\partial f}{\partial u} = f_1 - f_3 \mbox{, } \frac{\partial f}{\partial v} = f_2 - f_3. 
\end{multline*} 
 \begin{multline*}
        \frac{\partial f}{\partial x} = \frac{\partial f}{\partial u} \frac{\partial u}{\partial x} + \frac{\partial f}{\partial v} \frac{\partial v}{\partial x} = 
            \frac{1}{\Delta}(y_2 - y_3)(f_1 - f_3) - \frac{1}{\Delta}(y_1 - y_3)(f_2 - f_3) = \\
            \frac{1}{\Delta}\left[(y_2 - y_3)(f_1 - f_3) - (y_1 - y_3)(f_2 - f_3)\right]. 
    \end{multline*}
    \begin{multline*}
        \frac{\partial f}{\partial y} = \frac{\partial f}{\partial u} \frac{\partial u}{\partial y} + \frac{\partial f}{\partial v} \frac{\partial v}{\partial y} = 
            -\frac{1}{\Delta}(x_2 - y_3)(f_1 - f_3) + \frac{1}{\Delta}(x_1 - x_3)(f_2 - f_3) = \\
            \frac{1}{\Delta}\left[-(x_2 - x_3)(f_1 - f_3) + (x_1 - x_3)(f_2 - f_3)\right]. 
    \end{multline*}
    \begin{multline*}
        \left[\left(\frac{\partial f}{\partial x}\right)^2 + \left(\frac{\partial f}{\partial y}\right)^2 \right] = 
        \frac{1}{\Delta^2} 
        [
            (f_1 - f_3)^2 [(y_2 - y_3)^2 + (x_2 - x_3)^2] + \\ + (f_2 - f_3)^2 [(y_1 - y_3)^2 + (x_1 - x_3)^2] % - \\
            - 2 (f_1 - f_3)(f_2 - f_3)((y_2 - y_3)(y_1 - y_3) + (x_2 - x_3)(x_1 - x_3))
        ].
    \end{multline*}
    Введем обозначения: $\vect{AB} = \vect{c}$, $\vect{BC} = \vect{a}$, $\vect{CA} = \vect{b}$. Тогда предыдущую формулу можно записать следующим образом.
    \begin{multline*}
        \left[\left(\frac{\partial f}{\partial x}\right)^2 + \left(\frac{\partial f}{\partial y}\right)^2 \right] = 
        \frac{1}{\Delta^2} 
        \left[
            (f_1 - f_3)^2 a^2 + (f_2 - f_3)^2 b^2 - 2 (f_1 - f_3)(f_2 - f_3)(\vec{a}, \vec{b}) 
        \right] = \\
        = \frac{1}{\Delta^2} 
        \left[
            (f_1 - f_3)^2 a^2 + (f_2 - f_3)^2 b^2 - (f_1 - f_3)(f_2 - f_3)(a^2 + b^2 - c^2) 
        \right] = \\
        = \frac{1}{\Delta^2} 
        \left[
            a^2 (f_1 - f_3)^2 + b^2 (f_2 - f_3)^2 - (a^2 + b^2 - c^2)(f_1 f_2 - f_3(f_1 + f_2) + f_3^2)
        \right] = \\
        = \frac{1}{2 \Delta^2} 
        [
            2 a^2 (f_1 - f_3)^2 + 2 b^2 (f_2 - f_3)^2 - (a^2 + b^2 - c^2) \times \\
            \times ([2 f_1 f_2 - f_1^2 - f_2 ^ 2] + [f_1 ^2 - 2 f_3 f_1 + f_3 ^ 2] + [f_2^2 - 2 f_3 f_2 + f_3^2])
        ] = \\
      \end{multline*}
      \begin{multline*}
        = \frac{1}{2 \Delta^2} 
        [
            2 a^2 (f_1 - f_3)^2 + 2 b^2 (f_2 - f_3)^2 - 
            \left( a^2 + b^2 - c^2 \right) \times \\ \times \left(-(f_1 - f_2)^2 + (f_1 - f_3)^2 + (f_2 - f_3)^2 \right)
        ] = \\
        = \frac{1}{2 \Delta^2} 
        \left[
            (a^2 + b^2 - c^2)(f_1 - f_2)^2 + (a^2 + c^2 - b^2)(f_1 - f_3)^2 + (b^2 + c^2 - a^2)(f_2 - f_3)^2 
        \right] = \\
        = \frac{1}{\Delta^2} 
        \left[
            ab \cos{(\gamma)} (f_1 - f_2)^2 + ac \cos{(\beta)}(f_1 - f_3)^2 + bc \cos{(\alpha)}(f_2 - f_3)^2 
        \right] = \\
        = \frac{2S(T)}{\Delta^2} 
        \left[
            \cot{(\gamma)} (f_1 - f_2)^2 + \cot{(\beta)}(f_1 - f_3)^2 + \cot{(\alpha)}(f_2 - f_3)^2 
        \right].
    \end{multline*}
    \begin{multline*}
        \int_{T}{\left[\left(\frac{\partial f}{\partial x}\right)^2 + \left(\frac{\partial f}{\partial y}\right)^2 \right] dx dy} = \\
        = \int\limits_0^1\int\limits_0^{1-u} \frac{2S(T)}{\Delta^2} 
        \left[
            \cot{(\gamma)} (f_1 - f_2)^2 + \cot{(\beta)}(f_1 - f_3)^2 + \cot{(\alpha)}(f_2 - f_3)^2 
        \right] 
        |\Delta| du dv = \\
        = \frac{2S(T)}{|\Delta|} 
        \left[
            \cot{(\gamma)} (f_1 - f_2)^2 + \cot{(\beta)}(f_1 - f_3)^2 + \cot{(\alpha)}(f_2 - f_3)^2 
        \right]
        \int\limits_0^1\int\limits_0^{1-u} du dv = \\ 
        = \frac{S(T)}{|\Delta|} 
        \left[
            \cot{(\gamma)} (f_1 - f_2)^2 + \cot{(\beta)}(f_1 - f_3)^2 + \cot{(\alpha)}(f_2 - f_3)^2 
        \right].
    \end{multline*}
    Осталось заметить, что $|\Delta| = 2S(T)$. Окончательно получаем 
    $$
        \int_T ||\nabla{f}||^2 dT = \frac{1}{2} \left[ \cot{(\gamma)} (f_1 - f_2)^2 + \cot{(\beta)}(f_1 - f_3)^2 + \cot{(\alpha)}(f_2 - f_3)^2 \right] 
    $$
\begin{equation}
\label{formula:Pinkall}
  E_D(f) = \frac{1}{4} \sum_{T \in \mathfrak{T}} \left[ \cot{(\gamma)} (f_1 - f_2)^2 + \cot{(\beta)}(f_1 - f_3)^2 + \cot{(\alpha)}(f_2 - f_3)^2 \right].
\end{equation}

Формула~(\ref{formula:Pinkall}) может быть записана как 
\begin{equation}
\label{formula:EDOverEdges}
  E_D(f) = \frac{1}{4} \sum_{\left( i, j \right) \in E\left(\mathfrak{T}\right)}{w_{ij} |f_i - f_j|^2}, 
\end{equation}
где $E(\mathfrak{T})$ есть множество ребер триангуляции $\mathfrak{T}$, 
$w_{ij}~=~\cot\alpha_{ij}~+~\cot\alpha_{ji}$, $\alpha_{\left\{{ij, ji}\right\}}$~---~углы 
противолежащие ребру~$v_i v_j$ в смежных треугольках $\triangle T_1=v_i v_j v_{k_1}$, 
$\triangle T_2~=~v_j v_i v_{k_2}$. Если ребро~$v_i v_j$ принадлежит тольно одному треугольнику 
$v_i v_j v_k$ ($v_i v_j$ принадлежит границе $\Omega$), то $w_{ij} = \cot \alpha_{ij}$.
Если мы предположим, что $f$ --- комплексно-значная функция, то есть  
$$f_x=\re{f}=\frac{1}{2}\left(f + \conjugate{f}\right), f_y=\im{f}=\frac{1}{2}\left(f - \conjugate{f}\right),$$ 
то $$|f_i - f_j|^2 = \conjugate{(f_i - f_j)}(f_i - f_j) = (\conjugate{f_i} - \conjugate{f_j})(f_i - f_j).$$ 

Пусть $G(\mathfrak{T})$ обозначает взвешенный граф триангуляции $\mathfrak{T}$, $V(G) = V$, $E(G) = E(\mathfrak{T})$, вес ребра $(u, v) = w_{uv}$. 
Если рассмотреть лапласиан $L$ графа $G$ определенный следующим образом (\cite{Chen}) 
$$ L_{uv} = \begin{cases}
  \sum_{(v, x) \in E(G)}{w_{vx}}&\text{if $u = v,$} \\
  -w_{uv}&\text{if $u \sim v,$} \\
  0 &\text{otherwise;}
\end{cases}
$$ то формула~(\ref{formula:EDOverEdges}) будет эквивалентна следующей:
\begin{equation}
  \label{formula:EDOverLaplacian}
  E_D(f) = \frac{1}{4} \conjugate{\mathbf{f}^T} L \mathbf{f},
\end{equation}
где $\mathbf{f} = (f_1, f_2, \dots f_n)^T$, $\conjugate{\mathbf{f}^T} = 
(\conjugate{f_1}, \conjugate{f_2}, \dots, \conjugate{f_n})$. Заметим, что $L$ --- вещественный и 
симметричный, а значит, эрмитов.

$\int_{\Omega} \det(\nabla f) d\Omega$ также может быть выражен как некоторая 
эрмитова квадратичная форма, примененная к $\mathbf{f}$. 
Рассмотрим треугольник $\triangle T = v_i v_j v_k \in \mathfrak{T}$. 
$f$ линейна на $\triangle T$, следовательно $\det (\nabla f)$ есть константа, 
а следовательно, $\int_{\triangle T} \det (\nabla f) dT$ есть ориентированная площадь 
$\triangle T' = f_i f_j f_k$. 
Если предположить, что $v_i, v_j, v_k$ образуют левый поворот,~то
\begin{multline}
  \label{formula:det}
  \int_{\triangle T} \det (\nabla f) dT = \frac{1}{2} (f_i - f_k) \times (f_j - f_k) = \\
  = \frac{1}{2} \left[(f_i - f_k)_x (f_j - f_k)_y - (f_i - f_k)_y (f_j - f_k)_y \right] = \\
  = \frac{i}{4} \left[ (f_i - f_k)\conjugate{(f_j - f_k)} - (f_j - f_k) \conjugate{(f_i - f_k)} \right] = \\
  = \frac{i}{4} \left[ (f_i \conjugate{f_j} - \conjugate{f_i} f_j) + (f_j \conjugate{f_k} - \conjugate{f_j} f_k) + 
    (f_k \conjugate{f_i} - \conjugate{f_k} f_i) \right].  
\end{multline}
Подставляя (\ref{formula:det}) в формулу
\begin{equation}
  \label{formula:det_global}
  \int_{\Omega} \det {\left( \nabla f \right) } d\Omega = \sum_{\triangle T \in \mathfrak{T}}{ \int_{\triangle T} \det{ \left(\nabla f \right) } dT},
\end{equation}
можно заметить, что для каждого ребра $v_i v_j$ лежащего внутри выпуклой оболочки $\Omega$  
существует парное ребро $v_j v_i$, а следовательно, выражение 
$(f_i \conjugate{f_j}~-~\conjugate{f_i} f_j)$ содержится в правой части равенства (\ref{formula:det_global}) 
вместе с противоположным выражением $(f_j \conjugate{f_i}~-~\conjugate{f_j} f_i)$, то есть аннигилирует. 
Следовательно, формула (\ref{formula:det_global}) может быть переписана следующим образом
\begin{equation}
\label{formula:det_over_edges}
  \int_{\Omega} \det {\left( \nabla f \right) } d\Omega = \frac{i}{4} \sum_{v_i v_j \in \partial E}{(f_i \conjugate{f_j} - \conjugate{f_i} f_j)},
\end{equation}
где $\partial E = \partial \Omega \cap E(\mathfrak{T})$ --- множество ребер 
триангуляции $\mathfrak{T}$ принадлежащих границе $\Omega$. 
Если рассмотреть эрмитову матрицу $L'$ определенную следующим образом
\begin{equation*}
  L_{uv}' = \begin{cases}
    i  & \text{if edge $uv \in \partial E$ and $\Omega$ is left of the line $(u, v)$;} \\ 
    -i & \text{if edge $uv \in \partial E$ and $\Omega$ is right of the line $(u, v)$;} \\ 
    0  & \text{otherwise}
  \end{cases}
\end{equation*}
то формула~(\ref{formula:det_over_edges}) будет эквивалента следующией:
\begin{equation}
\label{formula:DetOverLaplacian}
  \int_{\Omega} \det {\left( \nabla f \right) } d\Omega = \frac{1}{4} \conjugate{\mathbf{f}^T} L' \mathbf{f}.
\end{equation}
Пользуясь формулами (\ref{formula:EC}), (\ref{formula:EDOverLaplacian}), 
(\ref{formula:DetOverLaplacian}) можно выразить $E_C(f)$ как значение некоторой квадратичной 
формы, примененной к $\mathbf{f}$
\begin{equation*}
  E_C(f) = \frac{1}{4}\conjugate{\mathbf{f}^T} L'' \mathbf{f},
\end{equation*}
где $L'' = L - L'$. Легко видеть, что оператор $L''$ --- эрмитов (как разность двух эрмитовых), 
следовательно, его собственные числа вещественны и из его из его собственных векторов можно составить 
ортонормированный базис~$\mathbb{C}^n$.

\section{Модификация базового алгоритма}
\label{sec:minimization}

Если $\mathbf{f} = \alpha \left[ p_1 \mathbf{e}_1 + p_2 \mathbf{e}_2 + \dots + p_k \mathbf{e}_k \right],$ 
где $\alpha$ --- некоторая положительная константа, $p_i \in \left\{-1, 1 \right\}$ и 
$\{\mathbf{e}_i\}_{i=1}^k$ --- ортонормированная система веторов, то 
$E_C(f) = \frac{1}{4}\conjugate{\mathbf{f}^T} L'' \mathbf{f}$ будет минимизировано, 
когда $\{\mathbf{e}_i\}_{i=1}^k$ суть собственные вектора $L''$ соответствующии $k$ наименьшим 
собственным числам. 

Базовый алгоритм (\ref{sec:base}) строил вектор $\mathbf{g}$ разниц между новыми и старыми
координатми вершин (формула (\ref{formula:g})), а $E_C$ --- функционал применяемый к $f$, 
$f(v) = v + g(v)$. Здесь нет противоречия, так как $$E_C(g) = E_C(f - v) = E_C(f),$$ ибо
функционал $E_C$ линейный и тождественная функция $id(v) = v$ конформна. 

Таким образом, чтобы минимизировать конформную энергию при использовании базового алгоритма 
необходимо выбрать вектора $\{\mathbf{h}_i\}_{i=1}^k$ из формулы (\ref{formula:g}) равными собстенным векторам $L''$, соответсвующим наименьшим собственным числам.

Конформная энергия была определена (\ref{sec:continious}) как мера искажения локальных углов.
Можно посмотреть, что выражает $E_C(f)$ в терминах углов триангуляции $\mathfrak{T}$. 
\begin{multline*}
  E_C(f) = E_D(f) - \int_{\Omega} \det (\nabla f) d\Omega = \\ 
  = \frac{1}{4} \sum_{v_i v_j v_k \in \mathfrak{T}} \cot{\alpha_{ij}}|f_i - f_j|^2 + 
  \cot{\alpha_{jk}}|f_j - f_k|^2 + \cot{\alpha_{ki}}|f_k - f_i|^2 - 4S(f_i f_j f_k),
\end{multline*}
где $S(f_i f_j f_k)$ есть ориентированная площать треугольника $\triangle f_i f_j f_k$. 
В предположении, что функция $f$ не сильно искажает область $\Omega$, 
ориентация треугольника $\triangle f_i f_j f_k$ равна ориентации треугольника 
$\triangle v_i v_j v_k$, то есть треугольник $\triangle f_i f_j f_k$ ориентирован против часовой
стрелки,
\begin{multline*}
  4 S(f_i f_j f_k) = \cot{\alpha_{ij}'}|f_i - f_j|^2 + \cot{\alpha_{jk}'}|f_j - f_k|^2 + \cot{\alpha_{ki}'}|f_k - f_i|^2,
\end{multline*}
где $\alpha_{ij}'$, $\alpha_{jk}'$, $\alpha_{ki}'$ --- углы, противолежащие сторонам $f_i f_j$, 
$f_j f_k$, $f_k f_i$ треугольника $\triangle f_i f_j f_k$ соответственно. 
Таким образом, при определенных предположениях
\begin{multline*}
  E_C(f) = \frac{1}{4} \sum_{v_i v_j v_k \in \mathfrak{T}} (\cot{\alpha_{ij}} - \cot{\alpha_{ij}'})|f_i - f_j|^2 + \\
  + (\cot{\alpha_{jk}} - \cot{\alpha_{jk}'})|f_j - f_k|^2 + (\cot{\alpha_{ki}} - \cot{\alpha_{ki}'})|f_k - f_i|^2,
\end{multline*} откуда видно, что подобные треугольники перешедшие при внедрении ЦВЗ в подобные,
дают вклад в меру искажения пропорциональный своей площади, что довольно естественно.

До этого момента на триангуляцию $\mathfrak{T}$ не накладывалось никаких условий, но очевидно, что
$E_C$ depends on $\mathfrak{T}$. Во-первых, было доказано, что минимизация $E_C$ влечет 
минимизацию углов триангуляции в целом, но наиболее важны с точки зрения визуального качества
карты угла между смежными ребрами исходных контуров и полилиний. Поэтому логично, 
добавить исходные контура и полилинии в триангуляцию $\mathfrak{T}$ как ограничения.
Во-вторых, при переходе от дискретного к непрерывному случаю триангуляция была использована для
построения кусочно-линейной интерполяционной поверхности, поэтому она должна быть выбрана так,
чтобы минимизировать погрешность интерполяции. Согласно теореме Rippa (\cite{Rippa}), 
триангуляция Делоне доставляет наименьшую погрешность КЛИП среди всех триангуляций 
множества точек.
Но так как для доказательства теоремы Rippa необходим только локальный критерий 
пустого круга(\cite{Chen}), она легко обобщается и на случай триангуляций с ограничениями

Таким образом, при построении сетки для вычисления частотного представления карты в 
базовом алгоритме (\ref{sec:base}) действительно имеет смысл строить триангуляцию Делоне 
с ограничениями. 


\section{Выводы к главе}
В данной главе была предложенна численная мера искажения исходных данным при внедрении в них ЦВЗ
и был модифицирован базовый алгоритм для минимизации предложенной меры.
