\documentclass{article}

\usepackage[T2A]{fontenc}
\usepackage[utf8]{inputenc}
\usepackage[english,russian]{babel}

\usepackage[a4paper,left=20mm,right=20mm,top=20mm,bottom=20mm]{geometry}
\begin{document}
\section{Алгоритм внедрения watermark'а}

\begin{itemize}
    \item На вход алгоритму подается облако точек.

    Экземпляр класса \texttt{watermarking::planar\_graph} [watermarking/common.h].

    \item Плоскость разбивается на прямоугольные ячейки с примерно одинаковым количеством вершин не превышающее определенную величину.

    \texttt{watermarking::embedding\_impl::subdivide\_plane} [watermarking/embedding.h] куда в качестве параметра \texttt{max\_subarea\_size} передается параметр конструктора \texttt{watermarking::embedding\_impl::embedding\_impl(...)} \texttt{max\_patch\_size}.

\end{itemize}

\end{document}
